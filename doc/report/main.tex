% ------------------------------ 76 characters -----------------------------
\documentclass{article}

% ------------------------------ 76 characters -----------------------------
\usepackage[l2tabu, orthodox]{nag}

% -- Texto e codificação
\usepackage[utf8]{inputenc}
\usepackage[portuguese]{babel}
\usepackage{lmodern}
\usepackage[T1]{fontenc}
\usepackage{microtype}
\usepackage{xcolor}

% -- Definir margens do documento
\usepackage{geometry}
\geometry{verbose, nomarginpar,
    tmargin = 2.5cm,
    bmargin = 2.5cm,
    lmargin = 2.5cm,
    rmargin = 2.5cm}
%\usepackage{showframe}

% -- Cabeçalho e rodapé
\usepackage{fancyhdr}
\fancyhf{}
\fancyhead[L]{\textsc{Instituto Superior Técnico}}
\fancyhead[R]{\textsc{Programação de Sistemas}}
\fancyfoot[C]{\thepage}
\pagestyle{fancy}

% -- Funções matemáticas extra
\usepackage{mathtools}
\usepackage{siunitx}

% -- Símbolos extra
\usepackage{amssymb}
\usepackage{textcomp}
\usepackage{gensymb}

% -- Referências
\usepackage{hyperref}

% --  Definições de imagens
\usepackage{graphicx}
\graphicspath{{graphics/}}
\usepackage{caption}
\usepackage{subcaption}

% -- Desenhar circuitos elétricos e lógicos
\usepackage{tikz}
\usepackage{pgfplots}
\usetikzlibrary{arrows.meta,positioning}
\pgfplotsset{compat=1.5}
\pgfplotsset{table/search path = {data}}
\pgfplotsset{	/pgf/number format/use comma,}

% -- Integrar código fonte
\usepackage{minted}

\usepackage{lipsum}

\begin{document}

\thispagestyle{empty}
	\includegraphics[viewport=9.5cm 11cm 0cm 0cm,scale=0.29]{IST_A_CMYK_POS}
	
	\begin{center}
		\vspace{70mm} % --  Espaço em branco
		\rule{\linewidth}{0.5pt} \\
		\vspace{2mm}
		\Huge \textsc{Clipboard Distribuído} \\
		\rule{\linewidth}{2pt} \\
		\vspace{8mm} % -- Espaço em branco
		\LARGE Projeto Final
		
		\vspace{\fill} % --  Espaço em branco variável
		\large
		Daniel de Schiffart \texttt{81479} \\
		João Gonçalves \texttt{81040}
		
		\vspace{10mm} % --  Espaço em branco
		\Large Instituto Superior Técnico \\
		\textit{Mestrado em Engenharia Aeroespacial} \\
		\vspace{1mm}
		\large Programação de Sistemas
		
		\vspace{10mm} % --  Espaço em branco
		\Large Junho de 2018
	\end{center}
\newpage

\tableofcontents

\newpage

\section{Introdução}

Para o projeto final da unidade curricular de Programação de Sistemas do
segundo semestre do ano curricular de 2017/2018 foi desenvolvido e
implementado um \textit{clipboard} distribuído em C. O objetivo do
projeto proposto consistia na implementação do ante-referido 
\textit{clipboard} através de uma biblioteca, uma API e um processo
local que controlassem o funcionamento do mesmo.

O funcionamento do \textit{clipboard} distribuído é análogo ao
funcionamento da função de copiar-colar implementada em muitos diferentes
programas e sistemas. Uma API permite a outras aplicações comunicar com
um processo local (o \textit{clipboard} local) e guardar e retirar dados
temporariamente, que serão sincronizados com outras instâncias do mesmo
\textit{clipboard} local que estão em funcionamento em outros sistemas,
desde que estejam conectados entre si. O conjunto das instâncias deste
\textit{clipboard} terá então os mesmos dados guardados em todos os
sistemas, facilitando assim a comunicação de dados entre estes.
Cada conjunto de \textit{clipboards} interligados tem espaço para
armazenamento de dez variáveis diferentes, identificados de 1 a 10.

As várias instâncias de \textit{clipboards} locais mantêm uma hierarquia
entre si. Quando um \text{clipboard} é iniciado, ou conecta-se a outro ou
servirá para outras instâncias se conectarem. Neste segundo caso, esta
primeira instância estará no topo da hierarquia, e quaisquer outras
instâncias que se conectem estarão um nível abaixo na hierarquia.

Para a comunicação entre \textit{clipboards} são usados \textit{sockets} de domínio Internet, enquanto que para a comunicação com clientes são usados sockets de domínio UNIX.

\begin{figure}[ht]
	\centering
	\scalebox{0.8}{
    \begin{tikzpicture}[clipboard/.style = {rectangle, draw, inner sep = 2cm},
    					app/.style = {rectangle, draw, inner sep = 3mm},
    					socketin/.style = {rectangle, color = black!70!green},
    					pathstyle/.style = {->},
    					pathlabel/.style ={color = black!40!blue}]
    	% Start of Clipboard 1
    	\node (app1) at (0,0) [app] {App 1};
    	\node (app2) [app, right = 1.7cm of app1] {App 2};
    	\node (t2app1) [socketin, below = 3cm of app1] {\texttt{thread\_1}};
    	\node (t2app2) [socketin, below = 3cm of app2] {\texttt{thread\_1}};
    	\draw [pathstyle] (app1) to [bend right = 20] node [auto,swap,align = right, pathlabel] {\footnotesize \ttfamily copy \\ \footnotesize \ttfamily paste \\ \footnotesize \ttfamily wait \\ \footnotesize \ttfamily disconnect} (t2app1);
    	\draw [pathstyle] (t2app1) to [bend right = 20] (app1);
    	\draw [pathstyle] (app2) to [bend right = 20] (t2app2);
    	\draw [pathstyle] (t2app2) to [bend right = 20] (app2);
    	\path (t2app1.north) to node (clip1) [auto, swap, clipboard] {\LARGE Clipboard 1} (t2app2.north);
    	\path (clip1.north east) to node (clip1in1) [auto, swap, align = right, yshift = 0.5cm, socketin] {\texttt{thread\_1}} node (clip1in3) [auto, swap, align = right, yshift = -0.5cm, socketin] {\texttt{thread\_3}} (clip1.south east);
    	% Start of Clipboard 2
    	\node (app3) at (10,0) [app] {App 3};
    	\node (app4) [app, right = 1.7cm of app3] {App 4};
    	\node (t2app3) [socketin, below = 3cm of app3] {\texttt{thread\_1}};
    	\node (t2app4) [socketin, below = 3cm of app4] {\texttt{thread\_1}};
    	\draw [pathstyle] (app3) to [bend right = 20] (t2app3);
    	\draw [pathstyle] (t2app3) to [bend right = 20] (app3);
    	\draw [pathstyle] (app4) to [bend right = 20] (t2app4);
    	\draw [pathstyle] (t2app4) to [bend right = 20] (app4);
    	\path (t2app3.north) to node (clip2) [auto, swap, clipboard] {\LARGE Clipboard 2} (t2app4.north);
    	\path (clip2.north west) to node (clip2in1) [auto, align = left, yshift = 0.5cm, socketin] {\texttt{thread\_1}} (clip2.south west);
    	% Paths in between
    	\draw [pathstyle] (clip1in1.north east) -- node [auto, align = center, pathlabel] {\ttfamily \footnotesize paste \\ \ttfamily \footnotesize redirect} (clip2in1.north west);
    	\draw [pathstyle] (clip2in1) -- (clip1in1);
    	\draw [pathstyle] (clip1in3.east) -- node [auto, swap, pathlabel] {\ttfamily \footnotesize redirect} (clip2in1);
    \end{tikzpicture}}
	\caption{Esboço da arquitectura de implementação e comunicação.}
	\label{fig:scheme_main}
\end{figure}

\begin{figure}[ht]
	\centering
    \begin{tikzpicture}[	clipinstance/.style = {circle,draw,
                                minimum width = 1.2cm},
    					    patharrow/.style = {-{Stealth[]}}]
    	\node (node1) at (0,0) [clipinstance] {\Large $1$};
    	\node (node2) [clipinstance, below right = 1.4cm and 0.8 cm of node1] {\Large $2$};
    	\node (node3) [clipinstance, below left = 1.4cm and 0.8 cm of node1] {\Large $3$};
    	\node (node4) [clipinstance, below right = 1.4cm and 0.8 cm of node2] {\Large $4$};
    	\node (node5) at (5,0) [clipinstance] {\Large $5$};
    	\node (node6) [clipinstance, below right = 1.4cm and 0.8 cm of node5] {\Large $6$};
    	\draw [patharrow] (node4) -- (node2) edge (node1);
    	\draw [patharrow] (node3) -- (node1);
    	\draw [patharrow] (node6) -- (node5);
    \end{tikzpicture}
	\caption{Hierarquia de instâncias do \textit{clipboard}.}
	\label{fig:hierarchy}
\end{figure}

\section{Arquitetura}

O programa desenvolvido encontra-se dividido num conjunto de cinco ficheiros
de código fonte, mais dois ficheiros adicionais desenvolvidos para duas
aplicações de teste. São estes ficheiros
\begin{itemize} \setlength{\itemsep}{0pt}
    \item \texttt{clipboard.c} -- O núcleo do \textit{clipboard}.
        Contém o código para a função principal \mintinline{c}{int main} e
        a \textit{thread} principal do programa, assim como a inclusão das
        bibliotecas utilizadas. \\
    \item \texttt{clipboard.h} e \texttt{clipboard-dev.h} -- Incluem a 
        declaração das funções e todos os macros utilizados.
        O primeiro ficheiro inclui as funções que compõem a API para
        implementação em outras aplicações, e o segundo ficheiro contém as
        declarações das funções restantes. \\
    \item \texttt{library.c} -- Contém o código das funções pertencentes à
        API, declaradas em \texttt{clipboard.h}.
    \item \texttt{clipboard\_func.c} -- Código auxiliar ao ficheiro
        \texttt{clipboard.c}. Inclui as funções correspondentes às
        \textit{threads} e ao tratamento de sinais, assim como outras funções
        utilizadas.
    \item \texttt{app\_teste.c} e \texttt{app\_teste2.c} -- Código
        correspondente às duas aplicações de teste referidas.
\end{itemize}

\subsection{Estruturas de dados}
\label{sec:estruturas}

Os dados são armazenados numa estrutura do tipo \mintinline{c}{struct clip_data}, declarada global. De \texttt{clipboard-dev.h}:
\begin{minted}[linenos, firstnumber = 43, frame = leftline]{c}
struct clip_data{
	
	void* data[10];
	size_t size[10];
	short waiting[10];
	short writing[10];
};
\end{minted}
Os últimos dois campos são usados para gerir os pedidos de \mintinline{c}{clipboard_wait()}.

Os \textit{peers}, têm o seu \textit{file descriptor} guardado num vetor dinâmico implementado com \mintinline{c}{struct clip_peers}.
\begin{minted}[linenos, firstnumber = 28, frame = leftline]{c}
typedef struct clip_peers{
	
	int count;
	int * sock;
	int master;
}C_peers;
\end{minted}
Nesta estrutura, guarda-se ainda o \texttt{fd} do \textit{master}, ou seja, o \textit{clipboard} conectado com a hierarquia mais alta, que é usado na sincronização global. 
Seguindo o exemplo da figura \ref{fig:hierarchy}, \texttt{master} do clipboard 4 vai guardar o \texttt{fd} de 2, e o \texttt{master} do clipboard 1 vai guardar $-1$.

\subsection{\textit{Threads}}

Cada instância de um \textit{clipboard} usa três tipos de \textit{threads} diferentes, mais o \texttt{main} em si, para executar todos os pedidos de forma eficiente e rápida.

\subsubsection{\texttt{main}} 
A função \texttt{main} efetua o arranque, que consiste na inicialização do \textit{clipboard}, sincronização deste (caso em modo conectado), lançamento das outras \textit{threads} relevantes, e de seguida entra num ciclo infinito para aceitar ligações com aplicações.


\subsubsection{\texttt{thread\_1}}
Serviço para aplicações ou outros \textit{clipboards}.
Esta \textit{thread} é lançada para cada nova conexão que é efetuada, e faz \textit{exit} quando esta é interrompida. 

As funcionalidades são:
\begin{itemize} \setlength{\itemsep}{0pt}
    \item Desconectar. \\
    \item \textit{Copy} -- escrever numa posição local, e reencaminhar para todos os \textit{peers} ligados a esta instância. \\
    \item \textit{Paste} -- devolver uma posição. \\
    \item \textit{Wait} -- esperar até uma posição ser alterada e depois devolve-la. \\
    \item \textit{Redirect} -- idêntico ao \textit{copy}, mas específico para usar com outros \textit{clipboards}.
\end{itemize}

\subsubsection{\texttt{thread\_2}}
Aceitação de ligações com outros \textit{clipboards}. Uma e uma só \textit{thread} deste tipo está a correr, em ciclo infinito, para aceitar ligações com \textit{peers}, lançando depois uma \texttt{thread\_1}. 
Também é aqui criada a socket de domínio \texttt{AF\_INET}, e imprimido o porto através o qual outro \textit{clipboard} se pode conectar.

\subsubsection{\texttt{thread\_3}}

Sincronização global. Igualmente à \texttt{thread\_2}, uma instância desta \textit{thread} corre por cada \textit{clipboard}. Caso o \textit{clipboard} seja o topo de uma árvore, a cada 30 segundos ocorre uma sincronização, de forma que todos os dados são replicados para os \textit{peers} a ele ligados. 

\subsection{API}

As funções referidas pertencentes à API são cinco e estão declaradas no
ficheiro \texttt{clipboard.h} da forma
\begin{minted}[linenos, firstnumber = 3]{c}
int clipboard_connect(char * clipboard_dir);
int clipboard_copy(int clipboard_id, int region, void *buf, size_t count);
int clipboard_paste(int clipboard_id, int region, void *buf, size_t count);
void clipboard_close(int clipboard_id);
int clipboard_wait(int clipboard_id, int region, void* buf, size_t count);
\end{minted}

\subsection{Protocolo de Comunicação}

A comunicação entre instâncias separadas do \textit{clipboard} é realizada
com recurso a \textit{sockets} UNIX do tipo \texttt{SOCK\_STREAM}.

O uso dos sockets é feito usando a estrutura \mintinline{c}{struct message} declarada em \texttt{clipboard-dev.h} como:
\begin{minted}[linenos, firstnumber = 19, frame = leftline]{c}
typedef struct message{
	
	short entry; 
	char msg[MESSAGE_SIZE];
	size_t size; 
	short flag; 
	
}Message;
\end{minted}
O macro \texttt{MESSAGE\_SIZE} tem o valor de 256 (bytes), que deverá assegurar a maioria dos pedidos ao sistema com apenas uma mensagem, para o uso típico deste. 
Caso isto se verifique, toda a informação necessária é transmitida em apenas uma mensagem: \texttt{entry} contêm a posição do \textit{clipboard} em questão, \texttt{size} o tamanho da mensagem, e \texttt{flag} é um código que distingue as operações a fazer.
Caso a mensagem tenha mais de \texttt{MESSAGE\_SIZE} bytes, o campo \texttt{size} passa a conter o tamanho dos dados que faltam receber, estes são divididos em partes, e enviados consecutivamente, como exemplificado neste segmento de código, proveniente da função \mintinline{c}{clipboard_copy}:
\begin{minted}[linenos, firstnumber = 61, frame = leftline]{c}
	while (count > MESSAGE_SIZE){
	
		m.size=count;
	
		memcpy(m.msg, buf, MESSAGE_SIZE);
		
		err = send(clipboard_id, &m, sizeof(Message),0);
\end{minted}
\begin{minted}[linenos, firstnumber = 72, frame = leftline]{c}
		sent+=MESSAGE_SIZE;
		buf = (char*)buf + MESSAGE_SIZE;
		count = count - MESSAGE_SIZE;
	}
	
	if (count > 0){
		m.size = count;
		memcpy(m.msg, buf, count);
		err = send(clipboard_id, &m, sizeof(Message),0);
\end{minted}
\begin{minted}[linenos, firstnumber = 85, frame = leftline]{c}
		sent+=count;
	}
\end{minted}
Do lado do receptor, a estrutura é idêntica.



\subsection{Fluxo de Tratamento de Pedidos}

A todas estas rotinas, acrescentam-se operações de alocação de memória e tratamento de erros.

\subsubsection{\textit{Copy}}

A função de \textit{copy} do \textit{clipboard} recebe uma mensagem e
sincroniza-a com todas as instâncias do \textit{clipboard} a que está
conectada.

Primeiro, os dados são totalmente copiados para um \textit{buffer} local, seguindo o protocolo de comunicação já descrito.

De seguida, os dados são inseridos na estrutura de memória do \textit{clipboard}, e verifica-se se a entrada está marcada como em espera, sendo efetuado um \mintinline{c}{pthread_cond_broadcast()} caso afirmativo.

Finalmente, retorna-se uma mensagem a reportar erros, e os dados são reencaminhados para todos os \textit{peers}.

\subsubsection{\textit{Paste}}

A função de \textit{paste} devolve a totalidade dos dados de uma entrada.

Começa por copiar os dados da entrada para um \textit{buffer} local, com o caso excecional de a entrada estar vazia.

De seguida, são enviados de acordo com o protocolo de comunicação.

\subsubsection{\textit{Wait}}

A função de \textit{wait} espera até uma entrada ser alterada para de seguida devolver os novos conteúdos.

A espera é sinalizada através de um campo da estrutura de dados do \textit{clipboard}, e entra-se em espera passiva, através de uma variável condicional.

Quando sinalizada de outra instrução de \textit{copy}, passa a executar como um \textit{paste}.


\section{Sincronização}
\subsection{Identificação das Regiões Críticas}

As regiões críticas a proteger são aquelas que procuram ler/escrever nas estruturas de dados descritas na secção \ref{sec:estruturas}. 

No caso de \mintinline{c}{struct clip_data}, os acessos ocorrem sempre que se faz um \textit{copy/redirect}, \textit{paste}, \textit{wait}, ou operação equivalente. 
A exclusão é efetuada com \texttt{mutexes}, de maneira que as leituras também bloqueiam os acessos.
Isto permite que tanto a escrita como a leitura sejam essencialmente atómicas, não havendo mistura de dados.
Como são usados \textit{buffers} locais para operações com os dados, a zona crítica pode ser restrita apenas a uma operação de alocamento de memória e de cópia.

No caso de \mintinline{c}{struct clip_peers}, os acessos ocorrem apenas quando há o reencaminhamento de dados, quer originado por um \textit{copy/redirect} ou por uma sincronização periódica, ou quando se insere ou remove um valor no vetor dinâmico. 

\subsection{Implementação de Exclusão Mútua}
De \texttt{clipboard.c}: 
\begin{minted}[linenos, firstnumber = 20, frame = leftline]{c}
//Secure data structures
pthread_mutex_t m_clip = PTHREAD_MUTEX_INITIALIZER;
pthread_mutex_t m_peers = PTHREAD_MUTEX_INITIALIZER;
\end{minted}
\subsubsection{\texttt{clip\_data}}

A exclusão mútua desta estrutura é feita usando o \texttt{mutex} \texttt{m\_clip}. De \texttt{clipboard\_func.c}:
\begin{minted}[linenos, firstnumber = 104, frame = leftline]{c}
//Critical region
pthread_mutex_lock(&m_clip);
			
//Check if we are rewriting the same data
if(clipboard.size[m.entry]==size)
	if(!memcmp(clipboard.data[m.entry], buf, size))
		err=1;
if (err != 1){	
	clipboard.data[m.entry] = 
			realloc(clipboard.data[m.entry], size);					
	if (clipboard.data[m.entry] == NULL){
		perror("realloc");
		exit(-1);
	}
	memcpy(clipboard.data[m.entry], buf, size);
	clipboard.size[m.entry]=size;
	print_entry(m.entry);
						
	//Signal waiting threads
	if (clipboard.waiting[m.entry] > 0){
		clipboard.writing[m.entry]=1;
		pthread_cond_broadcast(&cv_wait);
	}
}
pthread_mutex_unlock(&m_clip);
//End critical region	
\end{minted}

Este excerto é retirado de uma operação de \textit{copy}, onde os novos dados já estão guardados em \texttt{buf}. A zona crítica consiste numa alocação de memória, uma cópia, e uma sinalização a \textit{threads} em espera passiva. No caso de uma leitura, é ainda mais simples:

\begin{minted}[linenos, firstnumber = 202, frame = leftline]{c}
//Critical region
pthread_mutex_lock(&m_clip);
size=clipboard.size[m.entry];
if (size == 0){
	pthread_mutex_unlock(&m_clip);
\end{minted}
\begin{minted}[linenos, firstnumber = 219, frame = leftline]{c}
	continue;
}else{
	buf=realloc(buf, size);
	if(buf==NULL){
		perror("realloc");
		exit(-1);
	}
	memcpy(buf, clipboard.data[m.entry], size);		
	pthread_mutex_unlock(&m_clip);
}
//end critical region
\end{minted}

\subsubsection{\texttt{clip\_peers}}

Neste caso, a exclusão é mais apertada, visto querermos garantir que o protocolo de comunicação com várias mensagens não seja violado. Desta maneira, este vetor dinâmico é bloqueado durante toda a transmissão de dados entre \textit{peers}, para além das operações de escrita e leitura.
\begin{minted}[linenos, firstnumber = 147, frame = leftline]{c}
// Replicate to other clipboards - critical region
pthread_mutex_lock(&m_peers);
if(peers->count > 0){
	m.flag = REDIRECT;
	buf2=buf;
				
	for(i=0; i<peers->count; i++){
					
		if (client_fd != peers->sock[i]){
			printf("Sending to peer with fd %d\n", 
				peers->sock[i]);
						
			buf=buf2;
			m.size=size;
\end{minted}
Protocolo de comunicação
\begin{minted}[linenos, firstnumber = 191, frame = leftline]{c}
		}
	}
	buf=buf2;
}
//End critical region
pthread_mutex_unlock(&m_peers);
\end{minted}

A exclusão é implementada também quando pretendemos remover um \textit{peer} do vector, como exemplificado aqui.
\begin{minted}[linenos, firstnumber = 43, frame = leftline]{c}
err = recv(client_fd, &m, m_size, 0);
if(err <= 0){
	printf("Disconnected\n");
	close(client_fd);
	if(mode==PEER_SERVICE){
		pthread_mutex_lock(&m_peers);
		remove_fd(peers, client_fd);
		pthread_mutex_unlock(&m_peers);
	}
	pthread_exit(NULL);
}
\end{minted}
Também de maneira idêntica quando se acrescenta um elemento, na \texttt{thread\_2}.


\subsection{Sincronização global}

A sincronização global é aplicada para garantir que, na maioria do tempo, uma árvore de \textit{clipboards} contém a mesma informação. 
Imaginemos que no exemplo da figura \ref{fig:hierarchy}, executamos um pedido de \textit{copy} nos \textit{clipboards} 3 e 4, ao mesmo tempo. O sistema propaga as mensagens em ambas as direções na cadeia, e no final, uma metade pode ter dados diferentes da outra.

A solução implementada é a sincronização completa a cada 30 segundos, iniciada neste caso pelo \textit{clipboard} 1. 
Para tal, a \texttt{thread\_3} é acordada por um \texttt{SIGALARM}, verifica se é o topo da árvore e envia os dados.
O código relevante é todo o de \mintinline{c}{thread_3_handler()}. 


\section{Gestão de Recurso e Tratamento de Erros}

\subsection{Memória dinâmica}

Os arrays dinâmicos usados são operados com a função \mintinline{c}{realloc()}, de maneira que a memória usada é sempre a mínima possível.

Em caso de terminação do programa com \texttt{CTRL-C}, as posições do clipboard são igualmente libertadas. 
Contudo, os \textit{buffers} locais a algumas \textit{threads} não são, temos \textit{memory leak}.

\subsection{Outros recursos}

Os \textit{file descriptors} são libertados com \mintinline{c}{close()} sempre que é detetado que a ligação foi interrompida do outro lado, ou há um pedido de desconexão. 

Para as mesmas condições, as instâncias de \texttt{thread\_1} são terminadas com \mintinline{c}{pthread_exit()}, libertando os respetivos recursos. 
As restantes \textit{threads} são suposto correrem durante todo o uso do sistema, logo são terminadas quando o programa é terminado.


\subsection{Tratamento de erros}

Em muitos casos, usam-se funções ou \textit{system calls} cujo sucesso depende de aplicações não necessariamente confiáveis quanto à sua robustez, ou de processos a correr noutra máquina. 
Desta forma, procura-se que os erros esperados não afetem o sistema de forma significativa. 

\subsubsection{\textit{Sockets}}

A comunicação entre processos é feita com \textit{sockets}, usando as funções \mintinline{c}{send()} e \mintinline{c}{recv()}. 
O valor de retorno destas funções indica se houve um erro, e o procedimento adotado é, quando isto acontece, interromper a ligação com o cliente/peer, evidentemente não completando o seu pedido, e sair em ordem invocando \mintinline{c}{close()} e \mintinline{c}{pthread-exit()}.
A API foi usa o mesmo conceito, mas não fecha a ligação, passando essa decisão à aplicação através do valor de retorno.

Para o caso do \mintinline{c}{send()}, é necessário desativar o sinal \texttt{SIGPIPE}, o que é aceitável visto protegermos todas as invocações desta função. Como exemplo, de \texttt{clipboard\_func.c}:

\begin{minted}[linenos, firstnumber = 167, frame = leftline]{c}
err = send(peers->sock[i], &m,m_size,0);
if (err == -1){
	perror("send");
	remove_fd(peers, peers->sock[i]);
	close(peers->sock[i]);
	i--;
	continue;
}
\end{minted}

Outro exemplo é o caso do \mintinline{c}{accept()} ser terminado devido ao \texttt{SIGALARM}. Quando este é o caso, simplesmente voltamos a invocar esta função.
\begin{minted}[linenos, firstnumber = 381, frame = leftline]{c}
client_fd = accept(sock_fd, (struct sockaddr *) & client_addr, &size_addr);
if(client_fd == -1) {
	if (errno == EINTR)
		continue;
				
	perror("accept");
	exit(-1);
}
\end{minted}

No caso do \mintinline{c}{bind()}, faz-se \mintinline{c}{unlink()} ao endereço e repete-se a chamada. \texttt{clipboard.c}:
\begin{minted}[linenos, firstnumber = 180, frame = leftline]{c}
err = bind(sock_fd, (struct sockaddr *)&local_addr, sizeof(local_addr));
if(err == -1) {
	unlink(SOCK_ADDRESS);
	printf("Unlinked previous address.\n");
	err = bind(sock_fd, (struct sockaddr *)&local_addr, sizeof(local_addr));
	if(err == -1) {
		perror("bind");
		exit(-1);
	}
}
\end{minted}

Outros erros ocorridos na preparação dos \textit{sockets} são reportados e o programa termina.

\subsubsection{Outros erros}

O programa testa os valores de retorno das chamadas de \mintinline{c}{realloc()}, terminando em caso de erro.


\end{document}



%-------------------------------------------
%Não sei se isto se adequa a esta parte
%-------------------------------------------
\small
Como exemplo utilizaremos a função \texttt{clipboard\_connect}. Uma aplicação
desenvolvida começaria por invocar a função \texttt{clipboard\_connect}, que
tentará conectar com uma instância local do \textit{clipboard}. Dentro do
\textit{clipboard}, o pedido de conexão é recebido pela
\texttt{thread\_2\_handler}, e devolvido à função \texttt{thread\_1\_handler}
com uma variável do tipo \texttt{T\_param} (um \texttt{struct} pré-definido),
que contém os dados necessários. Este processo encontra-se representado pelas
linhas do ficheiro \texttt{clipboard\_func.c} que se encontram abaixo.
\begin{minted}[linenos, firstnumber = 353, frame = leftline]{c}
    sock_fd = socket(AF_INET, SOCK_STREAM, 0);
\end{minted}
\begin{minted}[linenos, firstnumber = 359, frame = leftline]{c}
    local_addr.sin_family = AF_INET;
    local_addr.sin_port = htons(port);
    local_addr.sin_addr.s_addr= INADDR_ANY;
    err = bind(sock_fd, (struct sockaddr *)&local_addr, sizeof(local_addr));
\end{minted}
\begin{minted}[linenos, firstnumber = 369, frame = leftline]{c}
    err = listen(sock_fd, 2);
\end{minted}
\begin{minted}[linenos, firstnumber = 377, frame = leftline]{c}
    client_fd = accept(sock_fd, (struct sockaddr *) & client_addr, &size_addr);
\end{minted}
\begin{minted}[linenos, firstnumber = 391, frame = leftline]{c}
		param.sock_id=client_fd;
		param.peers=peers;
		param.mode=0;
		
		pthread_create(&thread_id, NULL, thread_1_handler, &param);
\end{minted}

A \textit{thread} \texttt{thread\_1\_handler} será mais abaixo detalhada.
\normalsize
% ------------------------------ 76 characters -----------------------------
\documentclass{article}

% ------------------------------ 76 characters -----------------------------
\usepackage[l2tabu, orthodox]{nag}

% -- Texto e codificação
\usepackage[utf8]{inputenc}
\usepackage[portuguese]{babel}
\usepackage{lmodern}
\usepackage[T1]{fontenc}
\usepackage{microtype}
\usepackage{xcolor}

% -- Definir margens do documento
\usepackage{geometry}
\geometry{verbose, nomarginpar,
    tmargin = 2.5cm,
    bmargin = 2.5cm,
    lmargin = 2.5cm,
    rmargin = 2.5cm}
%\usepackage{showframe}

% -- Cabeçalho e rodapé
\usepackage{fancyhdr}
\fancyhf{}
\fancyhead[L]{\textsc{Instituto Superior Técnico}}
\fancyhead[R]{\textsc{Programação de Sistemas}}
\fancyfoot[C]{\thepage}
\pagestyle{fancy}

% -- Funções matemáticas extra
\usepackage{mathtools}
\usepackage{siunitx}

% -- Símbolos extra
\usepackage{amssymb}
\usepackage{textcomp}
\usepackage{gensymb}

% -- Referências
\usepackage{hyperref}

% --  Definições de imagens
\usepackage{graphicx}
\graphicspath{{graphics/}}
\usepackage{caption}
\usepackage{subcaption}

% -- Desenhar circuitos elétricos e lógicos
\usepackage{tikz}
\usepackage{pgfplots}
\usetikzlibrary{arrows.meta,positioning}
\pgfplotsset{compat=1.5}
\pgfplotsset{table/search path = {data}}
\pgfplotsset{	/pgf/number format/use comma,}

% -- Integrar código fonte
\usepackage{minted}

\begin{document}

\thispagestyle{empty}
	\includegraphics[viewport=9.5cm 11cm 0cm 0cm,scale=0.29]{IST_A_CMYK_POS}
	
	\begin{center}
		\vspace{70mm} % --  Espaço em branco
		\rule{\linewidth}{0.5pt} \\
		\vspace{2mm}
		\Huge \textsc{Clipboard Distribuído} \\
		\rule{\linewidth}{2pt} \\
		\vspace{8mm} % -- Espaço em branco
		\LARGE Projeto Final
		
		\vspace{\fill} % --  Espaço em branco variável
		\large
		Daniel de Schiffart \texttt{81479} \\
		João Gonçalves \texttt{81040}
		
		\vspace{10mm} % --  Espaço em branco
		\Large Instituto Superior Técnico \\
		\textit{Mestrado em Engenharia Aeroespacial} \\
		\vspace{1mm}
		\large Programação de Sistemas
		
		\vspace{10mm} % --  Espaço em branco
		\Large Junho de 2018
	\end{center}
\newpage

\section{Introdução}

Para o projeto final da unidade curricular de Programação de Sistemas do
segundo semestre do ano curricular de 2017/2018 foi desenvolvido e
implementado um \textit{clipboard} distribuído em C. O objetivo do
projeto proposto consistia na implementação do ante-referido 
\textit{clipboard} através de uma biblioteca, uma API e um processo
local que controlassem o funcionamento do mesmo.

O funcionamento do \textit{clipboard} distribuído é análogo ao
funcionamento da função de copiar-colar implementada em muitos diferentes
programas e sistemas. Uma API permite a outras aplicações comunicar com
um processo local (o \textit{clipboard} local) e guardar e retirar dados
temporariamente, que serão sincronizados com outras instâncias do mesmo
\textit{clipboard} local que estão em funcionamento em outros sistemas,
desde que estejam conectados entre si. O conjunto das instâncias deste
\textit{clipboard} terá então os mesmos dados guardados em todos os
sistemas, facilitando assim a comunicação de dados entre estes.
Cada conjunto de \textit{clipboards} interligados tem espaço para
armazenamento de dez variáveis diferentes, identificados de 1 a 10.

As várias instâncias de \textit{clipboards} locais mantêm uma hierarquia
entre si. Quando um \text{clipboard} é iniciado, ou conecta-se a outro ou
servirá para outras instâncias se conectarem. Neste segundo caso, esta
primeira instância estará no topo da hierarquia, e quaisquer outras
instâncias que se conectem estarão um nível abaixo na hierarquia.

\begin{figure}[ht]
	\centering
	\includegraphics[width = 0.4\linewidth]{example-grid-100x100pt}
	\caption{Esboço da arquitectura de implementação e comunicação.}
	\label{fig:scheme_main}
\end{figure}

\begin{figure}[ht]
	\centering
	\includegraphics[width = 0.4\linewidth]{example-grid-100x100pt}
	\caption{Hierarquia de instâncias do \textit{clipboard}.}
	\label{fig:hierarchy}
\end{figure}

\section{Arquitectura}

\subsection{Protocolo de Comunicação}

\subsection{Fluxo de Tratamento de Pedidos}

\subsubsection{\textit{Copy}}

\subsubsection{\textit{Paste}}

\subsubsection{\textit{Wait}}

\section{Sincronização}

\subsection{Identificação das Regiões Críticas}

\subsection{Implementação de Exclusão Mútua}

\section{Gestão de Recurso de Tratamento de Erros}

\end{document}
